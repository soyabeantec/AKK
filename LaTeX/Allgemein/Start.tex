\section*{Declaration of Academic Honesty}
Hereby, I declare that I have composed the presented paper independently on my own and without any other resources than the ones indicated. All thoughts taken directly or indirectly from external sources are properly denoted as such.

This paper has neither been previously submitted to another authority nor has it been published yet. \\[1em]
Leonding, \duedateen \\[5em]
\ifthenelse{\isundefined{\firstauthor}}{}{\firstauthor}
\ifthenelse{\isundefined{\secondauthor}}{}{\kern-1ex, \secondauthor}
\ifthenelse{\isundefined{\thirdauthor}}{}{\kern-1ex, \thirdauthor}
\ifthenelse{\isundefined{\fourthauthor}}{}{\kern-1ex, \fourthauthor} \\[5em]

\begin{otherlanguage}{german}
\section*{Eidesstattliche Erklärung}
Hiermit erkläre ich an Eides statt, dass ich die vorgelegte Diplomarbeit selbstständig und ohne Benutzung anderer als der angegebenen Hilfsmittel angefertigt habe. Gedanken, die aus fremden Quellen direkt oder indirekt übernommen wurden, sind als solche gekennzeichnet.

Die Arbeit wurde bisher in gleicher oder ähnlicher Weise keiner anderen Prüfungsbehörde vorgelegt und auch noch nicht veröffentlicht. \\[1em]
Leonding, am \duedatede \\[5em]
\ifthenelse{\isundefined{\firstauthor}}{}{\firstauthor}
\ifthenelse{\isundefined{\secondauthor}}{}{\kern-1ex, \secondauthor}
\ifthenelse{\isundefined{\thirdauthor}}{}{\kern-1ex, \thirdauthor}
\ifthenelse{\isundefined{\fourthauthor}}{}{\kern-1ex, \fourthauthor} \\[5em]
\end{otherlanguage}

\begin{otherlanguage}{english}
\begin{abstract}
In our thesis we have dedicated ourselves to the problem of data analytics, this means 
that we want to present knowledge gathered from a data source to support a 
data-driven decision making. This was developed with a prototype and which gave us our first experience with that kind of problem.
\newline
\newline
For this we have split the problem into three clearly defined areas of responsibility, as described below:

\begin{itemize}
	\item The extraction and transformation of data form a datasource.
	
	\item The loading, saving and serving of the gathered data.
	
	\item The insightful visualisation of the data.
\end{itemize}

\begin{flushright}
	\includegraphics[scale=.25]{images/Akkalytics.png}
\end{flushright}

\end{abstract}
\end{otherlanguage}

\begin{otherlanguage}{german}
\begin{abstract}
In unserer Diplomarbeit haben wir uns dem Problem der Datenanalyse gewidmet, mag heißen, dass wir aus einer Datenquelle und des daraus gewonnen Wissens, visuell dieses darstellen wollen, um einen rationalen datenbasierten Entscheidungsprozess zu unterstützen. Dies wurde im Rahmen eines Prototyps erarbeitet und damit wurden wir auch zum ersten Mal mit dieser Problemstellung konfrontiert.
\newline
\newline
Dazu haben wir diese Aufgabe in drei klare Aufgabenbereichen unterteilt, welche wie folgend lauten: 

\begin{itemize}
	\item Die Extraktion und Transformation der Daten aus einer Datenquelle.
	
	\item Das Laden, speichern und bereitstellen der erworbenen Daten.
	
	\item Die erkenntnisbringende Visualisierung der Daten.
\end{itemize}

\begin{flushright}
	\includegraphics[scale=.25]{images/Akkalytics.png}
\end{flushright}

\end{abstract}
\end{otherlanguage}

\section*{Danksagung}
Im Namen des AKK-Teams danken wir der MIC, dass wir mit Ihnen die Diplomarbeit durchführen durften. Insbesondere danken wir Bashar, der uns, so gut es ihm zeitlich möglich war, unterstützt hat. 
Auch wollen wir uns bei unserem Betreuer Prof. Aistleitner recht herzlich für die Zeit im vergangen Jahr bedanken. 


E mucha gracia a Zólen'. Pa'l tempo. Bo.